%% This file is part of Enblend.
%% Licence details can be found in the file COPYING.


\section{\label{sec:metadata}%
  \genidx{metadata}%
  \gensee{image metadata}{metadata}%
  Image Metadata}

\App{} transparently handles some image metadata.


\subsection{\label{sec:metadata-exif-iptc-xmp}%
  \genidx{metadata!EXIF@\acronym{EXIF}}%
  \gensee{EXIF@\acronym{EXIF}}{metadata, \acronym{EXIF}}%
  \genidx{metadata!IPTC@\acronym{IPTC}}%
  \gensee{IPTC@\acronym{IPTC}}{metadata, \acronym{IPTC}}%
  \genidx{metadata!XMP@\acronym{XMP}}%
  \gensee{XMP@\acronym{XMP}}{metadata, \acronym{XMP}}%
  \acronym{EXIF}, \acronym{IPTC}, and \acronym{XMP} Data}

\App{} automatically copies selected photo metadata, this is

\begin{itemize}
\item
  \acronym{EXIF} tags (see \uref{\exiforg}{Exchangeable Image Format} and
  \appendixName~\fullref{sec:metadata-handling}),

\item
  \acronym{IPTC} tags (see \uref{\iptcorg}{International Press Telecommunications Council}), and

\item
  \acronym{XMP} data (see \uref{\adobexmp}{Extensible Metadata Platform})
\end{itemize}

\noindent of the first input image to the output image.  The metadata in all other input files
is ignored.

See \sectionName~\ref{sec:compiled-in-features} on how to check whether your \App{} has been
compiled with this feature.


\subsection{\label{sec:metadata-icc}%
  \genidx{metadata!\acronym{ICC} color profile}%
  \gensee{ICC@\acronym{ICC} metadata}{metadata, \acronym{ICC}}%
  \acronym{ICC} Color Profiles}

For grayscale or color input images with \acronym{ICC} profiles attached (see also
\sectionName~\fullref{sec:image-requirements} on the consistency of \acronym{ICC} metadata
across the input images), \App{} honors the profiles in the whole \appisdoing{} process and
writes the output using \acronym{ICC} profile of the first input image and well as attaching it
to the output image.

\App{} has no provisions to force a particular \acronym{ICC} profile for the output image if the
input images contain profiles.  Use programs like for example
\flexipageref{\command{tificc}}{app:littlecms} to apply profiles to \acronym{TIFF} images.

For input images without color profiles see in particular
options~\flexipageref{\option{--fall\shyp back-pro\shyp file}}{opt:fallback-profile} and
\flexipageref{\option{--blend-col\shyp or\shyp space}}{opt:blend-colorspace}.

\chapterName~\fullref{sec:color-spaces} discusses \acronym{ICC} profiles and color spaces in
detail.


\subsection[Format Specific]{\label{sec:metadata-specific}%
  \genidx{metadata!image-format specific}%
  \gensee{image-format specific metadata}{metadata, image-format specific}%
  \gensee{format specific metadata}{metadata, image-format specific}%
  Image-Format Specific}

\genidx{metadata!image-format specific!pixel resolution}%
\gensee{pixel resolution}{metadata, image-format specific, pixel resolution}%
\gensee{resolution of pixels}{metadata, image-format specific, pixel resolution}%
\genidx{metadata!image-format specific!image offset}%
\gensee{image}{metadata, image-format specific, image offset}%
\gensee{offset of image}{metadata, image-format specific, image offset}%
Depending on the input-image type (\acronym{TIFF}, \acronym{PNG}, \acronym{BMP}, \dots) the
underlying image library, \uref{\hciiwrvigra}{\acronym{VIGRA}}, supplies both \App{} and
\OtherApp{} with some metadata most notably the $x$- and $y$-resolutions and the $x$- and
$y$-offset of the image with respect to the canvas.

\App{} always honors this kind of metadata if it is available and will warn of inconsistencies
in the input images.


%%% Local Variables:
%%% fill-column: 96
%%% End:
